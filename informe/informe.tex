\documentclass[a4paper,10pt,twoside]{article}

\usepackage[top=1in, bottom=1in, left=1in, right=1in]{geometry}
\usepackage[utf8]{inputenc}
\usepackage[spanish,es-ucroman,es-noquoting]{babel}
\usepackage{setspace}
\usepackage{fancyhdr}
\usepackage{lastpage}
\usepackage{amsmath}
\usepackage{amsfonts}
\usepackage{verbatim}
\usepackage{graphicx}
\usepackage{float}
\usepackage{algorithmic}
\usepackage{tikz}
\usetikzlibrary{calc}
\usetikzlibrary{decorations.pathreplacing}


% Evita que el documento se estire verticalmente para ocupar
% el espacio vacío en cada página.
\raggedbottom


%%%%%%%%%% Configuración de Fancyhdr - Inicio %%%%%%%%%%
\pagestyle{fancy}
\thispagestyle{fancy}
\lhead{Trabajo Práctico 2, Organización del Computador II}
\rhead{Belloli, Lovisolo, Petaccio}
\renewcommand{\footrulewidth}{0.4pt}
\cfoot{\thepage /\pageref{LastPage}}

\fancypagestyle{caratula} {
   \fancyhf{}
   \cfoot{\thepage /\pageref{LastPage}}
   \renewcommand{\headrulewidth}{0pt}
   \renewcommand{\footrulewidth}{0pt}
}
%%%%%%%%%% Configuración de Fancyhdr - Fin %%%%%%%%%%


\begin{document}


%%%%%%%%%%%%%%%%%%%%%%%%%%%%%%%%%%%%%%%%%%%%%%%%%%%%%%%%%%%%%%%%%%%%%%%%%%%%%%%
%% Carátula                                                                  %%
%%%%%%%%%%%%%%%%%%%%%%%%%%%%%%%%%%%%%%%%%%%%%%%%%%%%%%%%%%%%%%%%%%%%%%%%%%%%%%%


\thispagestyle{caratula}

\begin{center}

\includegraphics[height=2cm]{DC.png} 
\hfill
\includegraphics[height=2cm]{UBA.jpg} 

\vspace{2cm}

Departamento de Computación,\\
Facultad de Ciencias Exactas y Naturales,\\
Universidad de Buenos Aires

\vspace{4cm}

\begin{Huge}
Trabajo Práctico 3
\end{Huge}

\vspace{0.5cm}

\begin{Large}
Organización del Computador II
\end{Large}

\vspace{1cm}

Primer Cuatrimestre de 2013

\vspace{4cm}

Grupo: \textbf{Panceta y Mozzarella}

\vspace{0.5cm}

\begin{tabular}{|c|c|c|}
\hline
Apellido y Nombre & LU & E-mail\\
\hline
Laouen Louan Mayal Belloli  & 134/11 & lao.facu@gmail.com\\
Leandro Lovisolo      		& 645/11 & leandro@leandro.me\\
Lautaro José Petaccio 		& 443/11 & lausuper@gmail.com\\
\hline
\end{tabular}

\end{center}

\newpage


%%%%%%%%%%%%%%%%%%%%%%%%%%%%%%%%%%%%%%%%%%%%%%%%%%%%%%%%%%%%%%%%%%%%%%%%%%%%%%%
%% Índice                                                                    %%
%%%%%%%%%%%%%%%%%%%%%%%%%%%%%%%%%%%%%%%%%%%%%%%%%%%%%%%%%%%%%%%%%%%%%%%%%%%%%%%


\tableofcontents

\newpage


%%%%%%%%%%%%%%%%%%%%%%%%%%%%%%%%%%%%%%%%%%%%%%%%%%%%%%%%%%%%%%%%%%%%%%%%%%%%%%%
%% Introducción                                                              %%
%%%%%%%%%%%%%%%%%%%%%%%%%%%%%%%%%%%%%%%%%%%%%%%%%%%%%%%%%%%%%%%%%%%%%%%%%%%%%%%


\section{Introducción}

Bla bla bla

\section{Desarrollo}

\section{Preguntas y respuestas}
\subsection{¿Qué ocurre si se intenta escribir en la fila 26, columna 1 de la matriz de video, utilizando el segmento de la GDT que direcciona a la memoria de video? ¿Por
qué?}

\subsection{¿Qué ocurre si no se setean todos los registros de segmento al entrar en
modo protegido? ¿Es necesario setearlos todos? ¿Por qué?}
Si no se setean los registros al entrar en modo protegido corremos el riesgo de que, en algún momento intentemos utilizarlos y al no tener un selector correcto (existente, distinto del 0 y del segmento que deberíamos utilizar), obtengamos un GPE.

También hay que tener en cuenta de utilizar el selector correcto (código, nivel 0) a la hora de pasar a modo protegido.

Ejemplos de la situación:
\begin{enumerate}
	\item No haber seteado el registro de segmento de datos e intentar acceder a un dato
	\item No haber seteado el registro de segmento del stack y querer utilizarlo
\end{enumerate}

\subsection{¿Cómo se puede hacer para generar una excepción sin utilizar la instrucción int? Mencionar al menos 3 formas posibles.}
Otras formas de realizar una excepción:
\begin{enumerate}
	\item \textbf{General Protection Exception:}
	Ocurre por diversas acciones, por ejemplo, acceder a memoria fuera del segmento asignado.
	\item \textbf{Divide by 0:}
	Ocurre cuando se intenta realizar una división por 0.
	\item \textbf{Page fault:}
	Excepción relacionada con paginación, puede ocurrir al intentar acceder a memoria que no fue mapeada.
\end{enumerate}

\subsection{¿Cuáles son los valores del stack cuando se genera una interrupción? ¿Qué significan? Indicar para el caso de operar en nivel 3 y nivel 0}

\subsection{¿Puede el directorio de páginas estar en cualquier posición arbitraria de memoria?}

\subsection{¿Es posible acceder a una página de nivel de kernel desde usuario?}

\subsection{¿Se puede mapear una página física desde dos direcciones virtuales distintas, de manera tal que una esté mapeada con nivel de usuario y la otra a nivel de kernel? De ser posible, ¿Qué problemas puede traer?}

\subsection{¿Qué permisos pueden tener las tablas y directorios de páginas? ¿Cuáles
son los permisos efectivos de una dirección de memoria según los permisos del directorio
y tablas de páginas?}

\subsection{¿Es posible desde dos directorios de página, referenciar a una misma tabla de páginas?}

\subsection{¿Que es el TLB (Translation Lookaside Buffer) y para qué sirve?}

\subsection{¿Qué pasa si en la interrupción de teclado no se lee la tecla presionada?}

\subsection{¿Qué pasa si no se resetea el PIC?}

\subsection{Colocando un breakpoint luego de la cargar una tarea, ¿cómo se puede
verificar, utilizando el debugger de Bochs, que la tarea se cargó correctamente? ¿Cómo se
llega a esta conclusión?}

\subsection{¿Cómo se puede verificar si la conmutación de tarea fue exitosa?}
Si la conmutación de la tarea fue exitosa, esta comenzará a ejecutarse, pudiendose ver en el debugger de Bochs su ejecución. Además de esto, es posible verificar, utilizando \textbf{info tss} si la TSS fue cargada correctamente observando los valores correspondientes a la tarea.

\subsection{Se sabe que las tareas llaman a la interrupción 0x80 y por 0x90. ¿Qué ocurre si esta no está implementada? ¿Por qué?}


\end{document}